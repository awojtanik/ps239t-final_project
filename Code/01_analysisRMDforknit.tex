\documentclass[]{article}
\usepackage{lmodern}
\usepackage{amssymb,amsmath}
\usepackage{ifxetex,ifluatex}
\usepackage{fixltx2e} % provides \textsubscript
\ifnum 0\ifxetex 1\fi\ifluatex 1\fi=0 % if pdftex
  \usepackage[T1]{fontenc}
  \usepackage[utf8]{inputenc}
\else % if luatex or xelatex
  \ifxetex
    \usepackage{mathspec}
  \else
    \usepackage{fontspec}
  \fi
  \defaultfontfeatures{Ligatures=TeX,Scale=MatchLowercase}
\fi
% use upquote if available, for straight quotes in verbatim environments
\IfFileExists{upquote.sty}{\usepackage{upquote}}{}
% use microtype if available
\IfFileExists{microtype.sty}{%
\usepackage{microtype}
\UseMicrotypeSet[protrusion]{basicmath} % disable protrusion for tt fonts
}{}
\usepackage[margin=1in]{geometry}
\usepackage{hyperref}
\hypersetup{unicode=true,
            pdftitle={Final project - Wojtanik},
            pdfauthor={Wojtanik},
            pdfborder={0 0 0},
            breaklinks=true}
\urlstyle{same}  % don't use monospace font for urls
\usepackage{color}
\usepackage{fancyvrb}
\newcommand{\VerbBar}{|}
\newcommand{\VERB}{\Verb[commandchars=\\\{\}]}
\DefineVerbatimEnvironment{Highlighting}{Verbatim}{commandchars=\\\{\}}
% Add ',fontsize=\small' for more characters per line
\usepackage{framed}
\definecolor{shadecolor}{RGB}{248,248,248}
\newenvironment{Shaded}{\begin{snugshade}}{\end{snugshade}}
\newcommand{\AlertTok}[1]{\textcolor[rgb]{0.94,0.16,0.16}{#1}}
\newcommand{\AnnotationTok}[1]{\textcolor[rgb]{0.56,0.35,0.01}{\textbf{\textit{#1}}}}
\newcommand{\AttributeTok}[1]{\textcolor[rgb]{0.77,0.63,0.00}{#1}}
\newcommand{\BaseNTok}[1]{\textcolor[rgb]{0.00,0.00,0.81}{#1}}
\newcommand{\BuiltInTok}[1]{#1}
\newcommand{\CharTok}[1]{\textcolor[rgb]{0.31,0.60,0.02}{#1}}
\newcommand{\CommentTok}[1]{\textcolor[rgb]{0.56,0.35,0.01}{\textit{#1}}}
\newcommand{\CommentVarTok}[1]{\textcolor[rgb]{0.56,0.35,0.01}{\textbf{\textit{#1}}}}
\newcommand{\ConstantTok}[1]{\textcolor[rgb]{0.00,0.00,0.00}{#1}}
\newcommand{\ControlFlowTok}[1]{\textcolor[rgb]{0.13,0.29,0.53}{\textbf{#1}}}
\newcommand{\DataTypeTok}[1]{\textcolor[rgb]{0.13,0.29,0.53}{#1}}
\newcommand{\DecValTok}[1]{\textcolor[rgb]{0.00,0.00,0.81}{#1}}
\newcommand{\DocumentationTok}[1]{\textcolor[rgb]{0.56,0.35,0.01}{\textbf{\textit{#1}}}}
\newcommand{\ErrorTok}[1]{\textcolor[rgb]{0.64,0.00,0.00}{\textbf{#1}}}
\newcommand{\ExtensionTok}[1]{#1}
\newcommand{\FloatTok}[1]{\textcolor[rgb]{0.00,0.00,0.81}{#1}}
\newcommand{\FunctionTok}[1]{\textcolor[rgb]{0.00,0.00,0.00}{#1}}
\newcommand{\ImportTok}[1]{#1}
\newcommand{\InformationTok}[1]{\textcolor[rgb]{0.56,0.35,0.01}{\textbf{\textit{#1}}}}
\newcommand{\KeywordTok}[1]{\textcolor[rgb]{0.13,0.29,0.53}{\textbf{#1}}}
\newcommand{\NormalTok}[1]{#1}
\newcommand{\OperatorTok}[1]{\textcolor[rgb]{0.81,0.36,0.00}{\textbf{#1}}}
\newcommand{\OtherTok}[1]{\textcolor[rgb]{0.56,0.35,0.01}{#1}}
\newcommand{\PreprocessorTok}[1]{\textcolor[rgb]{0.56,0.35,0.01}{\textit{#1}}}
\newcommand{\RegionMarkerTok}[1]{#1}
\newcommand{\SpecialCharTok}[1]{\textcolor[rgb]{0.00,0.00,0.00}{#1}}
\newcommand{\SpecialStringTok}[1]{\textcolor[rgb]{0.31,0.60,0.02}{#1}}
\newcommand{\StringTok}[1]{\textcolor[rgb]{0.31,0.60,0.02}{#1}}
\newcommand{\VariableTok}[1]{\textcolor[rgb]{0.00,0.00,0.00}{#1}}
\newcommand{\VerbatimStringTok}[1]{\textcolor[rgb]{0.31,0.60,0.02}{#1}}
\newcommand{\WarningTok}[1]{\textcolor[rgb]{0.56,0.35,0.01}{\textbf{\textit{#1}}}}
\usepackage{graphicx,grffile}
\makeatletter
\def\maxwidth{\ifdim\Gin@nat@width>\linewidth\linewidth\else\Gin@nat@width\fi}
\def\maxheight{\ifdim\Gin@nat@height>\textheight\textheight\else\Gin@nat@height\fi}
\makeatother
% Scale images if necessary, so that they will not overflow the page
% margins by default, and it is still possible to overwrite the defaults
% using explicit options in \includegraphics[width, height, ...]{}
\setkeys{Gin}{width=\maxwidth,height=\maxheight,keepaspectratio}
\IfFileExists{parskip.sty}{%
\usepackage{parskip}
}{% else
\setlength{\parindent}{0pt}
\setlength{\parskip}{6pt plus 2pt minus 1pt}
}
\setlength{\emergencystretch}{3em}  % prevent overfull lines
\providecommand{\tightlist}{%
  \setlength{\itemsep}{0pt}\setlength{\parskip}{0pt}}
\setcounter{secnumdepth}{0}
% Redefines (sub)paragraphs to behave more like sections
\ifx\paragraph\undefined\else
\let\oldparagraph\paragraph
\renewcommand{\paragraph}[1]{\oldparagraph{#1}\mbox{}}
\fi
\ifx\subparagraph\undefined\else
\let\oldsubparagraph\subparagraph
\renewcommand{\subparagraph}[1]{\oldsubparagraph{#1}\mbox{}}
\fi

%%% Use protect on footnotes to avoid problems with footnotes in titles
\let\rmarkdownfootnote\footnote%
\def\footnote{\protect\rmarkdownfootnote}

%%% Change title format to be more compact
\usepackage{titling}

% Create subtitle command for use in maketitle
\providecommand{\subtitle}[1]{
  \posttitle{
    \begin{center}\large#1\end{center}
    }
}

\setlength{\droptitle}{-2em}

  \title{Final project - Wojtanik}
    \pretitle{\vspace{\droptitle}\centering\huge}
  \posttitle{\par}
    \author{Wojtanik}
    \preauthor{\centering\large\emph}
  \postauthor{\par}
      \predate{\centering\large\emph}
  \postdate{\par}
    \date{5/9/2020}


\begin{document}
\maketitle

\#Initial setup

\begin{Shaded}
\begin{Highlighting}[]
\CommentTok{\# remove all objects}
\KeywordTok{rm}\NormalTok{(}\DataTypeTok{list=}\KeywordTok{ls}\NormalTok{())}

\CommentTok{\# Load packages }
\KeywordTok{library}\NormalTok{(pacman)}
\end{Highlighting}
\end{Shaded}

\begin{verbatim}
## Warning: package 'pacman' was built under R version 3.6.3
\end{verbatim}

\begin{Shaded}
\begin{Highlighting}[]
\NormalTok{pacman}\OperatorTok{::}\KeywordTok{p\_unload}\NormalTok{(all)}
\end{Highlighting}
\end{Shaded}

\begin{verbatim}
## The following packages have been unloaded:
## pacman
\end{verbatim}

\begin{Shaded}
\begin{Highlighting}[]
\KeywordTok{library}\NormalTok{(pacman)}
\end{Highlighting}
\end{Shaded}

\begin{verbatim}
## Warning: package 'pacman' was built under R version 3.6.3
\end{verbatim}

\begin{Shaded}
\begin{Highlighting}[]
\KeywordTok{p\_load}\NormalTok{(tidyverse, data.table, rmarkdown, knitr, tinytex, magrittr,}
\NormalTok{       gplots, viridis, RColorBrewer, scales, grDevices, graphics,}
\NormalTok{       mapproj, ggmap, rgdal, tmap, maptools, tmaptools, foreign, skimr, lubridate}
\NormalTok{       )}

\CommentTok{\#This package helps me read in shapefiles}
\KeywordTok{library}\NormalTok{(}\StringTok{"sf"}\NormalTok{)}
\end{Highlighting}
\end{Shaded}

\begin{verbatim}
## Warning: package 'sf' was built under R version 3.6.3
\end{verbatim}

\begin{verbatim}
## Linking to GEOS 3.6.1, GDAL 2.2.3, PROJ 4.9.3
\end{verbatim}

\hypertarget{section}{%
\section{----}\label{section}}

\#PART ONE: Reading, Developing, and Cleaning the Data

\#First I read in my primary data of interest on provinces in Burkina
Faso \#The primary variable of interest (the DV) is the \# of defections
per province that occurred during a mass defection of 75+ party
officials who left the party in January 2014 and formed the basis for a
new opposition party, which was eventually swept to power after BF
President Compaore's ouster in October 2014. \#The data also includes a
dummy variable for whether or not defections occurred, as well as
province-level covariates, including ethnicity, language, urban/rural,
and some data from the 2008 Afrobarometer survey on \% of respondents
who reported support for democracy and \% of respondents who claimed
they did not trust the ruling party. \#The goal of the project is to (a)
display and map data on defections and (b) determine whether the
distribution of defections by provinces is correlated with any of these
province-level covariates.

\begin{Shaded}
\begin{Highlighting}[]
\KeywordTok{setwd}\NormalTok{(}\StringTok{"C:/Users/awojt/Documents/Berkeley/Classes Spring 2020/Computational Methods/ps239t{-}final{-}project/Data"}\NormalTok{)}
\NormalTok{provdata <{-}}\StringTok{ }\KeywordTok{read.csv}\NormalTok{(}\StringTok{"Province data.csv"}\NormalTok{)}
\end{Highlighting}
\end{Shaded}

\#Here's some specs on the data

\begin{Shaded}
\begin{Highlighting}[]
\KeywordTok{class}\NormalTok{(provdata)}
\end{Highlighting}
\end{Shaded}

\begin{verbatim}
## [1] "data.frame"
\end{verbatim}

\begin{Shaded}
\begin{Highlighting}[]
\CommentTok{\#Province variable (unit of analysis)}
\KeywordTok{class}\NormalTok{(provdata}\OperatorTok{$}\NormalTok{ï..province)}
\end{Highlighting}
\end{Shaded}

\begin{verbatim}
## [1] "factor"
\end{verbatim}

\begin{Shaded}
\begin{Highlighting}[]
\CommentTok{\#Defector data (outcome variable)}
\KeywordTok{class}\NormalTok{(provdata}\OperatorTok{$}\NormalTok{defector\_number)}
\end{Highlighting}
\end{Shaded}

\begin{verbatim}
## [1] "integer"
\end{verbatim}

\begin{Shaded}
\begin{Highlighting}[]
\CommentTok{\#Sample of covariates}
\KeywordTok{class}\NormalTok{(provdata}\OperatorTok{$}\NormalTok{primary\_ethnicity)}
\end{Highlighting}
\end{Shaded}

\begin{verbatim}
## [1] "factor"
\end{verbatim}

\begin{Shaded}
\begin{Highlighting}[]
\KeywordTok{class}\NormalTok{(provdata}\OperatorTok{$}\NormalTok{percent\_demsupport2008)}
\end{Highlighting}
\end{Shaded}

\begin{verbatim}
## [1] "numeric"
\end{verbatim}

\begin{Shaded}
\begin{Highlighting}[]
\CommentTok{\#Running dim reveals the data has 45 rows (provinces) and 9 columns (variables)}
\KeywordTok{dim}\NormalTok{(provdata)}
\end{Highlighting}
\end{Shaded}

\begin{verbatim}
## [1] 45  9
\end{verbatim}

\begin{Shaded}
\begin{Highlighting}[]
\CommentTok{\#The primary variable of interest is the \# of defectors by province. A basic histogram reveals that most provinces had 1{-}2 defectors, but some had far more)}
\KeywordTok{hist}\NormalTok{(provdata}\OperatorTok{$}\NormalTok{defector\_number)}
\end{Highlighting}
\end{Shaded}

\includegraphics{01_analysisRMDforknit_files/figure-latex/unnamed-chunk-3-1.pdf}

\begin{Shaded}
\begin{Highlighting}[]
\CommentTok{\#Viewing the data reveals that the provinces with the highest \# of defectors were Kadiogo, Yatenga, and Sanmatenga}
\KeywordTok{View}\NormalTok{(provdata)}
\CommentTok{\#Here\textquotesingle{}s a summary of all the variables}
\KeywordTok{summary}\NormalTok{(provdata)}
\end{Highlighting}
\end{Shaded}

\begin{verbatim}
##      ï..province               region   defector_number  binary_defector 
##  Bale      : 1   Boucle de Mouhoun: 6   Min.   : 0.000   Min.   :0.0000  
##  Bam       : 1   Est              : 5   1st Qu.: 0.000   1st Qu.:0.0000  
##  Banwa     : 1   Centre-Ouest     : 4   Median : 1.000   Median :1.0000  
##  Bazega    : 1   Nord             : 4   Mean   : 1.511   Mean   :0.6889  
##  Bougouriba: 1   Sahel            : 4   3rd Qu.: 2.000   3rd Qu.:1.0000  
##  Boulgou   : 1   Sud-Ouest        : 4   Max.   :10.000   Max.   :1.0000  
##  (Other)   :39   (Other)          :18                                    
##      urban         primary_language  primary_ethnicity
##  Min.   :0.0000   Moore    :22      Mossi     :22     
##  1st Qu.:0.0000   Gulmacera: 3      Gourmatche: 3     
##  Median :0.0000   Bissa    : 2      Bissa     : 2     
##  Mean   :0.1111   Bobo     : 2      Bobo      : 2     
##  3rd Qu.:0.0000   Dagari   : 2      Dagari    : 2     
##  Max.   :1.0000   Fulfude  : 2      Gourounsi : 2     
##                   (Other)  :12      (Other)   :12     
##  percent_demsupport2008 percent_donottrust2008
##  Min.   :25.00          Min.   : 6.25         
##  1st Qu.:50.00          1st Qu.:20.00         
##  Median :62.50          Median :31.25         
##  Mean   :60.97          Mean   :29.94         
##  3rd Qu.:75.00          3rd Qu.:38.12         
##  Max.   :90.00          Max.   :57.14         
##  NA's   :5              NA's   :5
\end{verbatim}

\#The provdata dataset did not fall magically out of the sky. It had to
be created by merging data from a variety of sources. \#The first is
data on 68 of the 75 defectors for which I have data on their home
province (I lacked data on home province on 7 of them). This data comes
from a signed letter by the 75 defectors when they announced their
resignation: \url{https://lefaso.net/spip.php?article57333} . \#From
this I created a new csv file with data on the first and last names of
the defectors and their reported province

\begin{Shaded}
\begin{Highlighting}[]
\CommentTok{\#See this data below}
\KeywordTok{setwd}\NormalTok{(}\StringTok{"C:/Users/awojt/Documents/Berkeley/Classes Spring 2020/Computational Methods/ps239t{-}final{-}project/Data"}\NormalTok{)}
\NormalTok{defectordata <{-}}\StringTok{ }\KeywordTok{read.csv}\NormalTok{(}\StringTok{"defector\_data.csv"}\NormalTok{)}
\end{Highlighting}
\end{Shaded}

\#Some specs on the data

\begin{Shaded}
\begin{Highlighting}[]
\KeywordTok{head}\NormalTok{(defectordata)}
\end{Highlighting}
\end{Shaded}

\begin{verbatim}
##     ï..first_name last_name   Province
## 1   Flakie Victor     Shafo       Bale
## 2          Yimeon  Sawadogo        Bam
## 3          Drissa    Konate Bougouriba
## 4    Abdoul Karim  Ouattara Bougouriba
## 5 Domba Jean Marc      Palm Bougouriba
## 6       Z. Maxime      Some Bougouriba
\end{verbatim}

\begin{Shaded}
\begin{Highlighting}[]
\KeywordTok{class}\NormalTok{(defectordata)}
\end{Highlighting}
\end{Shaded}

\begin{verbatim}
## [1] "data.frame"
\end{verbatim}

\begin{Shaded}
\begin{Highlighting}[]
\KeywordTok{class}\NormalTok{(defectordata}\OperatorTok{$}\NormalTok{Province)}
\end{Highlighting}
\end{Shaded}

\begin{verbatim}
## [1] "factor"
\end{verbatim}

\begin{Shaded}
\begin{Highlighting}[]
\KeywordTok{dim}\NormalTok{(defectordata)}
\end{Highlighting}
\end{Shaded}

\begin{verbatim}
## [1] 75  3
\end{verbatim}

\begin{Shaded}
\begin{Highlighting}[]
\KeywordTok{View}\NormalTok{(defectordata)}
\KeywordTok{summary}\NormalTok{(defectordata) }\CommentTok{\#P.S. Ouedraogo is one of the most common last names in BF; this also shows that Kadiogo has the most defectors (and there are 7 whose province is unknown)}
\end{Highlighting}
\end{Shaded}

\begin{verbatim}
##        ï..first_name     last_name        Province 
##  Abdoulaye    : 2    Ouedraogo:10   Kadiogo   :10  
##  Abdoul Karim : 1    Sawadogo : 6             : 7  
##  Aboua        : 1    Some     : 4   Yatenga   : 7  
##  Alassane     : 1    Ouattara : 3   Sanmatenga: 5  
##  Alassane Bala: 1    Hien     : 2   Bougouriba: 4  
##  Amand R.P.   : 1    Konate   : 2   Boulkiemde: 4  
##  (Other)      :68    (Other)  :48   (Other)   :38
\end{verbatim}

\#From this defector data, I created a column in the main province data
(provdata) titled defector\_number, which adds the number of defectors
by province.

\begin{Shaded}
\begin{Highlighting}[]
\CommentTok{\#Here we see that there were 68 total defectors.}
\KeywordTok{sum}\NormalTok{(provdata}\OperatorTok{$}\NormalTok{defector\_number)}
\end{Highlighting}
\end{Shaded}

\begin{verbatim}
## [1] 68
\end{verbatim}

\begin{Shaded}
\begin{Highlighting}[]
\CommentTok{\#Summary stats reveal the mean, median, and quartiles for the defection data}
\KeywordTok{summary}\NormalTok{(provdata}\OperatorTok{$}\NormalTok{defector\_number)}
\end{Highlighting}
\end{Shaded}

\begin{verbatim}
##    Min. 1st Qu.  Median    Mean 3rd Qu.    Max. 
##   0.000   0.000   1.000   1.511   2.000  10.000
\end{verbatim}

\begin{Shaded}
\begin{Highlighting}[]
\CommentTok{\#A histogram reveals that most provinces had only 1{-}2 defectors, but some (e.g., Kadiogo, Yatenga, and Sanmatenga) had a lot.}
\KeywordTok{hist}\NormalTok{(provdata}\OperatorTok{$}\NormalTok{defector\_number)}
\end{Highlighting}
\end{Shaded}

\includegraphics{01_analysisRMDforknit_files/figure-latex/unnamed-chunk-6-1.pdf}

\#The province data also includes several additional columns, including:

\begin{Shaded}
\begin{Highlighting}[]
\CommentTok{\#1) province: the names of the 45 provinces in BF}
\KeywordTok{head}\NormalTok{(provdata}\OperatorTok{$}\NormalTok{ï..province)}
\end{Highlighting}
\end{Shaded}

\begin{verbatim}
## [1] Bale       Bam        Banwa      Bazega     Bougouriba Boulgou   
## 45 Levels: Bale Bam Banwa Bazega Bougouriba Boulgou Boulkiemde ... Zoundweogo
\end{verbatim}

\begin{Shaded}
\begin{Highlighting}[]
\CommentTok{\#2) region: the region (next largest administrative unit) for each province}
\KeywordTok{summary}\NormalTok{(provdata}\OperatorTok{$}\NormalTok{region)}
\end{Highlighting}
\end{Shaded}

\begin{verbatim}
## Boucle de Mouhoun          Cascades            Centre        Centre-Est 
##                 6                 2                 1                 3 
##       Centre-Nord      Centre-Ouest        Centre-Sud               Est 
##                 3                 4                 3                 5 
##     Hauts-Bassins              Nord   Plateau-Central             Sahel 
##                 3                 4                 3                 4 
##         Sud-Ouest 
##                 4
\end{verbatim}

\begin{Shaded}
\begin{Highlighting}[]
\CommentTok{\#3) binary\_defector: a binary variable that reveals whether or not the province experienced defections}
\KeywordTok{summary}\NormalTok{(provdata}\OperatorTok{$}\NormalTok{binary\_defector)}
\end{Highlighting}
\end{Shaded}

\begin{verbatim}
##    Min. 1st Qu.  Median    Mean 3rd Qu.    Max. 
##  0.0000  0.0000  1.0000  0.6889  1.0000  1.0000
\end{verbatim}

\begin{Shaded}
\begin{Highlighting}[]
\CommentTok{\#4) urban: another dichotomous variable coded whether the province is urban/rural. (It is coded as urban if it has a city of at least 50,000, rural otherwise)}
\KeywordTok{sum}\NormalTok{(provdata}\OperatorTok{$}\NormalTok{urban)}
\end{Highlighting}
\end{Shaded}

\begin{verbatim}
## [1] 5
\end{verbatim}

\begin{Shaded}
\begin{Highlighting}[]
\CommentTok{\#5) primary\_language: factor variable for the predominant language spoken in the province (assembled from Afrobarometer data and other sources)}
\KeywordTok{summary}\NormalTok{(provdata}\OperatorTok{$}\NormalTok{primary\_language)}
\end{Highlighting}
\end{Shaded}

\begin{verbatim}
##     Bissa      Bobo    Dagari   Fulfude Gourounsi Gulmacera  Karaboro 
##         2         2         2         2         2         3         1 
##      Lobi     Marka     Moore      Samo   Senoufo    Sonrai     Yanan 
##         2         2        22         1         2         1         1
\end{verbatim}

\begin{Shaded}
\begin{Highlighting}[]
\CommentTok{\#6) primary\_ethnicity: factor variable for predominant ethnicity of residents in provinces (ibid)}
\KeywordTok{summary}\NormalTok{(provdata}\OperatorTok{$}\NormalTok{primary\_ethnicity)}
\end{Highlighting}
\end{Shaded}

\begin{verbatim}
##      Bissa       Bobo     Dagari Gourmatche  Gourounsi   Karaboro 
##          2          2          2          3          2          1 
##       Lobi      Marka      Mossi       Peul       Samo    Senoufo 
##          2          2         22          2          1          2 
##     Sonrai      Yanan 
##          1          1
\end{verbatim}

\begin{Shaded}
\begin{Highlighting}[]
\CommentTok{\#7) percent\_demsupport2008: integer data on the \% of residents who claimed that they supported democracy from the 2008 Afrobarometer survey data (the latest survey with province{-}level data before the 2014 defection)}
\CommentTok{\#Note that I constructed this variable by merging and cleaning data from the original survey (see below)}
\KeywordTok{summary}\NormalTok{(provdata}\OperatorTok{$}\NormalTok{percent\_demsupport2008)}
\end{Highlighting}
\end{Shaded}

\begin{verbatim}
##    Min. 1st Qu.  Median    Mean 3rd Qu.    Max.    NA's 
##   25.00   50.00   62.50   60.97   75.00   90.00       5
\end{verbatim}

\begin{Shaded}
\begin{Highlighting}[]
\CommentTok{\#Note that 5 provinces were not surveyed so are recorded as missing}
\CommentTok{\#8) percent\_donottrust2008: integer data on the \% of residents who claimed in the 2008 Afrobarometer survey that they did not trust the ruling party}
\KeywordTok{summary}\NormalTok{(provdata}\OperatorTok{$}\NormalTok{percent\_demsupport2008)}
\end{Highlighting}
\end{Shaded}

\begin{verbatim}
##    Min. 1st Qu.  Median    Mean 3rd Qu.    Max.    NA's 
##   25.00   50.00   62.50   60.97   75.00   90.00       5
\end{verbatim}

\begin{Shaded}
\begin{Highlighting}[]
\CommentTok{\#Note that I also constructed this variable by merging and cleaning data from the original survey (see below)}
\end{Highlighting}
\end{Shaded}

\hypertarget{section-1}{%
\section{-----}\label{section-1}}

\#PART ONE ADDENDUM: How I constructed the percent\_demsupport2008 and
percent\_donottrust2008 variables

\#I constructed the two Afrobarometer variables by adapting them from
the original survey data \#First I read in the data:

\begin{Shaded}
\begin{Highlighting}[]
\KeywordTok{library}\NormalTok{(haven)}
\KeywordTok{setwd}\NormalTok{(}\StringTok{"C:/Users/awojt/Documents/Berkeley/Classes Spring 2020/Computational Methods/ps239t{-}final{-}project/Data"}\NormalTok{)}
\NormalTok{afro2008data <{-}}\StringTok{ }\KeywordTok{read.spss}\NormalTok{(}\StringTok{"bfo\_r4\_data.sav"}\NormalTok{, }\DataTypeTok{to.data.frame =} \OtherTok{TRUE}\NormalTok{)}
\end{Highlighting}
\end{Shaded}

\begin{verbatim}
## Warning in read.spss("bfo_r4_data.sav", to.data.frame = TRUE):
## bfo_r4_data.sav: Unrecognized record type 7, subtype 8 encountered in
## system file
\end{verbatim}

\begin{verbatim}
## re-encoding from CP1252
\end{verbatim}

\begin{verbatim}
## Warning in read.spss("bfo_r4_data.sav", to.data.frame = TRUE): Undeclared
## level(s) 18, 19, 20, 21, 22, 23, 24, 25, 26, 27, 28, 29, 30, 31, 32, 33,
## 34, 35, 36, 37, 38, 39, 40, 41, 42, 43, 44, 45, 46, 47, 48, 49, 50, 51, 52,
## 53, 54, 55, 56, 57, 58, 59, 60, 61, 62, 63, 64, 65, 66, 67, 68, 69, 70, 71,
## 72, 73, 75, 76, 77, 78, 80, 84, 85, 87, 89, 90, 95, 99 added in variable:
## Q1
\end{verbatim}

\begin{verbatim}
## Warning in read.spss("bfo_r4_data.sav", to.data.frame = TRUE): Undeclared
## level(s) 1, 2, 3, 4, 5, 6, 7 added in variable: Q88F
\end{verbatim}

\begin{verbatim}
## Warning in read.spss("bfo_r4_data.sav", to.data.frame = TRUE): Undeclared
## level(s) 1, 2, 3, 4, 5, 6, 7, 8, 9, 10, 12, 13, 15, 20, 22, 40, 50 added in
## variable: Q96
\end{verbatim}

\begin{verbatim}
## Warning in read.spss("bfo_r4_data.sav", to.data.frame = TRUE): Undeclared
## level(s) 25, 26, 27, 28, 30, 32, 33, 34, 35, 36, 37, 38, 39, 40, 41, 42,
## 43, 44, 45, 46, 47, 48, 49, 50, 51, 52, 53, 54, 55, 56, 57, 58, 59, 60, 61,
## 62, 63, 64, 65, 66, 67, 68, 69, 70, 71, 72, 73, 74, 75, 76, 77, 78, 79, 80,
## 81, 82, 83, 84, 85, 86, 87, 88, 89, 90, 91, 92, 93, 94, 95, 96, 97, 98, 99,
## 100, 101, 102, 103, 104, 105, 106, 107, 108, 109, 110, 111, 114, 115, 118,
## 119, 120, 121, 122, 123, 124, 125, 126, 128, 129, 133, 135, 137, 138, 140,
## 143, 145, 147, 150, 154, 178, 184, 211, 224, 485 added in variable: LENGTH
\end{verbatim}

\begin{verbatim}
## Warning in read.spss("bfo_r4_data.sav", to.data.frame = TRUE): Undeclared
## level(s) 22, 24, 26, 27, 28, 29, 30, 31, 32, 43 added in variable: Q111
\end{verbatim}

\#Then I kept only the questions that I wanted, created a new dataset
titled afro2008datanew

\begin{Shaded}
\begin{Highlighting}[]
\NormalTok{afro2008dataupdate <{-}}\StringTok{ }\KeywordTok{subset}\NormalTok{(afro2008data, }\DataTypeTok{select =} \KeywordTok{c}\NormalTok{(REGION, PROVINCE, DISTRICT, Q3, Q29A, Q29B, Q29C, Q30, Q31, Q32, Q34, Q35, Q37, Q38, Q40A, Q40B, Q42A, Q42B, Q42C, Q42D, Q43, Q44A, Q45B, Q49A, Q49E, Q49F, Q79, Q88E, Q89, Q90, Q97, Q101, Q102)) }
\KeywordTok{View}\NormalTok{(afro2008dataupdate)}
\end{Highlighting}
\end{Shaded}

\#From here I created some new variables using mutate()

\begin{Shaded}
\begin{Highlighting}[]
\NormalTok{afro2008dataupdate }\OperatorTok{\%<>\%}
\StringTok{  }\CommentTok{\#produces new logical variable based on Q30 (do you support democracy?)}
\StringTok{  }\KeywordTok{mutate}\NormalTok{(}\DataTypeTok{demsupport =} \KeywordTok{ifelse}\NormalTok{(Q30 }\OperatorTok{==}\StringTok{ "STATEMENT 1: Democracy preferable."}\NormalTok{, T, F)) }\OperatorTok{\%>\%}
\StringTok{  }\CommentTok{\#converts Q89 (what is highest level of education you have received?) to numeric}
\StringTok{  }\KeywordTok{mutate}\NormalTok{(}\DataTypeTok{education =} \KeywordTok{as.numeric}\NormalTok{(Q89)) }\OperatorTok{\%>\%}
\StringTok{  }\CommentTok{\#converts Q49E (how much do you trust the ruling party?) into numeric}
\StringTok{  }\KeywordTok{mutate}\NormalTok{(}\DataTypeTok{rulingpartytrust =} \KeywordTok{as.numeric}\NormalTok{(Q49E))}
\end{Highlighting}
\end{Shaded}

\hypertarget{then-i-checked-for-missing-data}{%
\section{Then I checked for missing
data}\label{then-i-checked-for-missing-data}}

\begin{Shaded}
\begin{Highlighting}[]
\CommentTok{\# Remove white space}
\NormalTok{afro2008dataupdate }\OperatorTok{\%<>\%}
\StringTok{  }\KeywordTok{mutate\_if}\NormalTok{(is.character, }\KeywordTok{list}\NormalTok{(str\_trim))}
\CommentTok{\# Check for empty cells}
\NormalTok{afro2008dataupdate }\OperatorTok{\%>\%}
\StringTok{  }\CommentTok{\# Keep only character variables }
\StringTok{  }\KeywordTok{select\_if}\NormalTok{(is.character) }\OperatorTok{\%>\%}
\StringTok{  }\CommentTok{\# Recode character variables as 1 if cell is empty }
\StringTok{  }\KeywordTok{mutate\_all}\NormalTok{(}\KeywordTok{list}\NormalTok{(}\OperatorTok{\textasciitilde{}}\KeywordTok{ifelse}\NormalTok{(.}\OperatorTok{==}\StringTok{""}\NormalTok{,}\DecValTok{1}\NormalTok{,}\DecValTok{0}\NormalTok{))) }\OperatorTok{\%>\%}
\StringTok{  }\CommentTok{\# Add up empty cells for each character variable }
\StringTok{  }\KeywordTok{summarise\_all}\NormalTok{(sum, }\DataTypeTok{na.rm=}\NormalTok{T) }\OperatorTok{\%>\%}
\StringTok{  }\CommentTok{\# Transpose data for visibility }
\StringTok{  }\KeywordTok{t}\NormalTok{()}
\end{Highlighting}
\end{Shaded}

\begin{verbatim}
##      [,1]
\end{verbatim}

\#Then I generated summary statistics for the variables of interest,
sorted by province:

\begin{Shaded}
\begin{Highlighting}[]
\CommentTok{\# I found the mean of the "rulingpartytrust" variable by province to get a sense of where the ruling party is more/less trusted.}
\NormalTok{afro2008dataupdate }\OperatorTok{\%>\%}
\StringTok{  }\KeywordTok{group\_by}\NormalTok{(PROVINCE) }\OperatorTok{\%>\%}
\StringTok{  }\KeywordTok{summarise}\NormalTok{(}\DataTypeTok{rulingpartytrust\_mean =} \KeywordTok{mean}\NormalTok{(rulingpartytrust, }\DataTypeTok{na.rm=}\NormalTok{T))}
\end{Highlighting}
\end{Shaded}

\begin{verbatim}
## # A tibble: 40 x 2
##    PROVINCE                         rulingpartytrust_mean
##    <fct>                                            <dbl>
##  1 "BAM                           "                  3.94
##  2 "BANWA                         "                  4.35
##  3 "BAZEGA                        "                  4.75
##  4 "BOULGOU                       "                  4.58
##  5 "BOULKIEMDE                    "                  4.22
##  6 "COMOE                         "                  4.1 
##  7 "GANZOURGOU                    "                  5.1 
##  8 "GNAGNA                        "                  4.75
##  9 "GOURMA                        "                  5   
## 10 "HOUET                         "                  3.85
## # ... with 30 more rows
\end{verbatim}

\begin{Shaded}
\begin{Highlighting}[]
\CommentTok{\# I also found the sum of total respondents who claim they supported democracy (699)}
\NormalTok{afro2008dataupdate }\OperatorTok{\%>\%}
\StringTok{  }\KeywordTok{summarise}\NormalTok{(}\KeywordTok{sum}\NormalTok{(demsupport}\OperatorTok{==}\NormalTok{T))}
\end{Highlighting}
\end{Shaded}

\begin{verbatim}
##   sum(demsupport == T)
## 1                  699
\end{verbatim}

\hypertarget{i-then-generated-the-of-respondents-who-claim-they-supported-democracy-for-each-province.}{%
\section{I then generated the \# of respondents who claim they supported
democracy for each
province.}\label{i-then-generated-the-of-respondents-who-claim-they-supported-democracy-for-each-province.}}

\begin{Shaded}
\begin{Highlighting}[]
\NormalTok{afro2008dataupdate }\OperatorTok{\%>\%}
\StringTok{  }\KeywordTok{group\_by}\NormalTok{(PROVINCE) }\OperatorTok{\%>\%}
\StringTok{  }\KeywordTok{summarise}\NormalTok{(}\KeywordTok{sum}\NormalTok{(demsupport}\OperatorTok{==}\NormalTok{T))}
\end{Highlighting}
\end{Shaded}

\begin{verbatim}
## # A tibble: 40 x 2
##    PROVINCE                         `sum(demsupport == T)`
##    <fct>                                             <int>
##  1 "BAM                           "                     12
##  2 "BANWA                         "                     27
##  3 "BAZEGA                        "                     11
##  4 "BOULGOU                       "                     28
##  5 "BOULKIEMDE                    "                     36
##  6 "COMOE                         "                     20
##  7 "GANZOURGOU                    "                     17
##  8 "GNAGNA                        "                     14
##  9 "GOURMA                        "                      6
## 10 "HOUET                         "                     42
## # ... with 30 more rows
\end{verbatim}

\#Then I found the \% of respondents who claimed they support democracy
for each province. This became the basis for the percent\_demsupport2008
variable in the main province data!

\begin{Shaded}
\begin{Highlighting}[]
\NormalTok{percentdemsupport <{-}}\StringTok{ }\NormalTok{afro2008dataupdate }\OperatorTok{\%>\%}
\StringTok{  }\KeywordTok{group\_by}\NormalTok{(PROVINCE) }\OperatorTok{\%>\%}
\StringTok{  }\KeywordTok{summarise}\NormalTok{((}\KeywordTok{sum}\NormalTok{(demsupport}\OperatorTok{==}\NormalTok{T)}\OperatorTok{/}\KeywordTok{n}\NormalTok{())}\OperatorTok{*}\DecValTok{100}\NormalTok{)}
\KeywordTok{print}\NormalTok{(percentdemsupport)}
\end{Highlighting}
\end{Shaded}

\begin{verbatim}
## # A tibble: 40 x 2
##    PROVINCE                         `(sum(demsupport == T)/n()) * 100`
##    <fct>                                                         <dbl>
##  1 "BAM                           "                               75  
##  2 "BANWA                         "                               67.5
##  3 "BAZEGA                        "                               68.8
##  4 "BOULGOU                       "                               70  
##  5 "BOULKIEMDE                    "                               90  
##  6 "COMOE                         "                               50  
##  7 "GANZOURGOU                    "                               42.5
##  8 "GNAGNA                        "                               43.8
##  9 "GOURMA                        "                               25  
## 10 "HOUET                         "                               52.5
## # ... with 30 more rows
\end{verbatim}

\#I did the same for the \% of respondents who do not ``trust'' the
ruling party, by province (respondents who claimed that they trusted the
ruling party ``not at all'' OR ``just a little'', meaning a response of
2 or 3 in my new rulingpartytrust variable) \#This became the
percent\_donottrust2008 variable in provdata!

\begin{Shaded}
\begin{Highlighting}[]
\NormalTok{percentdonottrust <{-}}\StringTok{ }\NormalTok{afro2008dataupdate }\OperatorTok{\%>\%}
\StringTok{  }\KeywordTok{group\_by}\NormalTok{(PROVINCE) }\OperatorTok{\%>\%}
\StringTok{  }\KeywordTok{summarise}\NormalTok{((}\KeywordTok{sum}\NormalTok{(rulingpartytrust}\OperatorTok{<}\DecValTok{4}\NormalTok{)}\OperatorTok{/}\KeywordTok{n}\NormalTok{())}\OperatorTok{*}\DecValTok{100}\NormalTok{)}
\KeywordTok{print}\NormalTok{(percentdonottrust)}
\end{Highlighting}
\end{Shaded}

\begin{verbatim}
## # A tibble: 40 x 2
##    PROVINCE                         `(sum(rulingpartytrust < 4)/n()) * 100`
##    <fct>                                                              <dbl>
##  1 "BAM                           "                                    31.2
##  2 "BANWA                         "                                    22.5
##  3 "BAZEGA                        "                                    18.8
##  4 "BOULGOU                       "                                    22.5
##  5 "BOULKIEMDE                    "                                    22.5
##  6 "COMOE                         "                                    32.5
##  7 "GANZOURGOU                    "                                    10  
##  8 "GNAGNA                        "                                    15.6
##  9 "GOURMA                        "                                    12.5
## 10 "HOUET                         "                                    51.2
## # ... with 30 more rows
\end{verbatim}

\#For fun, I could also calculate the percentage of Burkinabe who do not
``trust'' the ruling party based on education levels:

\begin{Shaded}
\begin{Highlighting}[]
\NormalTok{afro2008dataupdate }\OperatorTok{\%>\%}
\StringTok{  }\KeywordTok{group\_by}\NormalTok{(Q89) }\OperatorTok{\%>\%}
\StringTok{  }\KeywordTok{summarise}\NormalTok{(}\KeywordTok{sum}\NormalTok{(rulingpartytrust}\OperatorTok{<}\DecValTok{4}\NormalTok{)}\OperatorTok{/}\KeywordTok{n}\NormalTok{())}
\end{Highlighting}
\end{Shaded}

\begin{verbatim}
## # A tibble: 9 x 2
##   Q89                                         `sum(rulingpartytrust < 4)/n~
##   <fct>                                                               <dbl>
## 1 No formal schooling                                                 0.244
## 2 Informal schooling only                                             0.293
## 3 Some primary schooling                                              0.351
## 4 Primary school completed                                            0.371
## 5 Some secondary school/high school                                   0.521
## 6 Secondary school completed/high school                              0.591
## 7 Post-secondary qualifications, not univers~                         0.875
## 8 Some university                                                     0.536
## 9 University completed                                                0.6
\end{verbatim}

\#This suggests that better educated people have less trust in the
ruling party!

\hypertarget{section-2}{%
\section{------}\label{section-2}}

\#PART TWO: Data Visualization I (Graphs and Plots)

\#set dimensions and load themes

\begin{Shaded}
\begin{Highlighting}[]
\NormalTok{knitr}\OperatorTok{::}\NormalTok{opts\_chunk}\OperatorTok{$}\KeywordTok{set}\NormalTok{(}\DataTypeTok{fig.width=}\DecValTok{12}\NormalTok{, }\DataTypeTok{fig.height=}\DecValTok{8}\NormalTok{)}
\KeywordTok{source}\NormalTok{(}\StringTok{"C:/Users/awojt/Documents/Berkeley/Classes Spring 2020/Computational Methods/PS239T\_Spring2020 new/PS239T\_Spring2020/09\_r{-}analysis{-}visualization/06\_setup/visualisation.R"}\NormalTok{)}
\KeywordTok{library}\NormalTok{(ggplot2)}
\end{Highlighting}
\end{Shaded}

\#In this section, I set out to create some basic bivariate and
multivariate plots to display some of my data \#First, I plot the \# of
defectors per province (note that I flip the x and y-axis (h/t Julia)
and change the labels and scale):

\begin{Shaded}
\begin{Highlighting}[]
\NormalTok{plotbyprovince <{-}}\StringTok{ }\KeywordTok{ggplot}\NormalTok{(}\DataTypeTok{data =}\NormalTok{ provdata, }\KeywordTok{aes}\NormalTok{(}\DataTypeTok{x =} \KeywordTok{reorder}\NormalTok{(ï..province, defector\_number), }\DataTypeTok{y =}\NormalTok{ defector\_number)) }\OperatorTok{+}
\StringTok{  }\KeywordTok{geom\_bar}\NormalTok{(}\DataTypeTok{fill=}\StringTok{"purple"}\NormalTok{, }\DataTypeTok{stat=}\StringTok{"identity"}\NormalTok{) }\OperatorTok{+}\StringTok{ }\KeywordTok{labs}\NormalTok{(}\DataTypeTok{title=}\StringTok{"Defectors by Province, Jan 2014"}\NormalTok{, }\DataTypeTok{x=}\StringTok{"Province"}\NormalTok{, }\DataTypeTok{y=}\StringTok{"Number of Defectors"}\NormalTok{) }\OperatorTok{+}\StringTok{ }\KeywordTok{coord\_flip}\NormalTok{() }\OperatorTok{+}\StringTok{ }\KeywordTok{scale\_y\_continuous}\NormalTok{(}\DataTypeTok{breaks =} \KeywordTok{c}\NormalTok{(}\DecValTok{0}\OperatorTok{:}\DecValTok{10}\NormalTok{))}
\NormalTok{plotbyprovince}
\end{Highlighting}
\end{Shaded}

\includegraphics{01_analysisRMDforknit_files/figure-latex/unnamed-chunk-18-1.pdf}

\#Second, I test the \# of defections against a few covariates. For
example, I plot the \# of defectors by primary ethnicity:

\begin{Shaded}
\begin{Highlighting}[]
\NormalTok{plotbyethnicity <{-}}\StringTok{ }\KeywordTok{ggplot}\NormalTok{(}\DataTypeTok{data =}\NormalTok{ provdata, }\KeywordTok{aes}\NormalTok{(}\DataTypeTok{x =}\NormalTok{ primary\_ethnicity, }\DataTypeTok{y =}\NormalTok{ defector\_number)) }\OperatorTok{+}
\StringTok{  }\KeywordTok{geom\_bar}\NormalTok{(}\DataTypeTok{fill=}\StringTok{"red"}\NormalTok{, }\DataTypeTok{stat=}\StringTok{"identity"}\NormalTok{) }\OperatorTok{+}\StringTok{ }\KeywordTok{labs}\NormalTok{(}\DataTypeTok{title=}\StringTok{"Defectors by Ethnicity, Jan 2014"}\NormalTok{, }\DataTypeTok{x=}\StringTok{"Primary Ethnicity of Home Province"}\NormalTok{, }\DataTypeTok{y=}\StringTok{"Number of Defectors"}\NormalTok{) }\OperatorTok{+}\StringTok{ }\KeywordTok{scale\_y\_continuous}\NormalTok{(}\DataTypeTok{breaks =} \KeywordTok{c}\NormalTok{(}\DecValTok{5}\NormalTok{, }\DecValTok{10}\NormalTok{, }\DecValTok{15}\NormalTok{, }\DecValTok{20}\NormalTok{, }\DecValTok{25}\NormalTok{, }\DecValTok{30}\NormalTok{, }\DecValTok{35}\NormalTok{, }\DecValTok{40}\NormalTok{, }\DecValTok{45}\NormalTok{))}
\NormalTok{plotbyethnicity}
\end{Highlighting}
\end{Shaded}

\includegraphics{01_analysisRMDforknit_files/figure-latex/unnamed-chunk-19-1.pdf}
\#Here we see that ethnic Mossi account for most of the defectors. This
is not surprising, however, because it is the largest ethnic group in
Burkina Faso. \#We see a similar trend with primary\_language, with
Moore speakers (the language of the Mossi) as most prominent:

\begin{Shaded}
\begin{Highlighting}[]
\NormalTok{plotbylanguage <{-}}\StringTok{ }\KeywordTok{ggplot}\NormalTok{(}\DataTypeTok{data =}\NormalTok{ provdata, }\KeywordTok{aes}\NormalTok{(}\DataTypeTok{x =}\NormalTok{ primary\_language, }\DataTypeTok{y =}\NormalTok{ defector\_number)) }\OperatorTok{+}
\StringTok{  }\KeywordTok{geom\_bar}\NormalTok{(}\DataTypeTok{fill=}\StringTok{"blue"}\NormalTok{, }\DataTypeTok{stat=}\StringTok{"identity"}\NormalTok{) }\OperatorTok{+}\StringTok{ }\KeywordTok{labs}\NormalTok{(}\DataTypeTok{title=}\StringTok{"Defectors by Primary Language, Jan 2014"}\NormalTok{, }\DataTypeTok{x=}\StringTok{"Primary Language of Home Province"}\NormalTok{, }\DataTypeTok{y=}\StringTok{"Number of Defectors"}\NormalTok{) }\OperatorTok{+}\StringTok{ }\KeywordTok{scale\_y\_continuous}\NormalTok{(}\DataTypeTok{breaks =} \KeywordTok{c}\NormalTok{(}\DecValTok{5}\NormalTok{, }\DecValTok{10}\NormalTok{, }\DecValTok{15}\NormalTok{, }\DecValTok{20}\NormalTok{, }\DecValTok{25}\NormalTok{, }\DecValTok{30}\NormalTok{, }\DecValTok{35}\NormalTok{, }\DecValTok{40}\NormalTok{, }\DecValTok{45}\NormalTok{))}
\NormalTok{plotbylanguage}
\end{Highlighting}
\end{Shaded}

\includegraphics{01_analysisRMDforknit_files/figure-latex/unnamed-chunk-20-1.pdf}

\#Third, I look at defections against the ruling party support data in a
multivariate plot.

\begin{Shaded}
\begin{Highlighting}[]
\CommentTok{\#First I create a new logical variable (defectors\_yes\_or\_no) to stand{-}in for whether a province had defectors or not.}
\NormalTok{provdata }\OperatorTok{\%<>\%}
\StringTok{  }\KeywordTok{mutate}\NormalTok{(}\DataTypeTok{Defections =} \KeywordTok{ifelse}\NormalTok{(binary\_defector}\OperatorTok{==}\DecValTok{1}\NormalTok{, T, F))}

\CommentTok{\#Then I use this new variable as the "color" for the bar graph that maps \% of respondents who do not trust ruling party by province}
\NormalTok{plotbyrpsupport <{-}}\StringTok{ }\KeywordTok{ggplot}\NormalTok{(}\DataTypeTok{data =}\NormalTok{ provdata, }\KeywordTok{aes}\NormalTok{(}\DataTypeTok{x =} \KeywordTok{reorder}\NormalTok{(ï..province, percent\_donottrust2008), }\DataTypeTok{y =}\NormalTok{ percent\_donottrust2008, }\DataTypeTok{by =}\NormalTok{ percent\_donottrust2008, }\DataTypeTok{fill =}\NormalTok{ Defections, }\DataTypeTok{color =}\NormalTok{ Defections)) }\OperatorTok{+}\StringTok{ }\KeywordTok{geom\_bar}\NormalTok{(}\DataTypeTok{colour=}\StringTok{"white"}\NormalTok{, }\DataTypeTok{stat=}\StringTok{"identity"}\NormalTok{) }\OperatorTok{+}\StringTok{ }\KeywordTok{labs}\NormalTok{(}\DataTypeTok{title=} \StringTok{"Levels of Disapproval of Ruling Party by Province"}\NormalTok{, }\DataTypeTok{x=}\StringTok{"Province"}\NormalTok{, }\DataTypeTok{y=}\StringTok{"\% of Respondents Who Do Not Trust Ruling Party"}\NormalTok{) }\OperatorTok{+}\StringTok{ }\KeywordTok{coord\_flip}\NormalTok{() }\OperatorTok{+}\StringTok{ }\KeywordTok{scale\_y\_continuous}\NormalTok{(}\DataTypeTok{breaks =} \KeywordTok{c}\NormalTok{(}\DecValTok{5}\NormalTok{, }\DecValTok{10}\NormalTok{, }\DecValTok{15}\NormalTok{, }\DecValTok{20}\NormalTok{, }\DecValTok{25}\NormalTok{, }\DecValTok{30}\NormalTok{, }\DecValTok{35}\NormalTok{, }\DecValTok{40}\NormalTok{, }\DecValTok{45}\NormalTok{, }\DecValTok{50}\NormalTok{, }\DecValTok{55}\NormalTok{, }\DecValTok{60}\NormalTok{))}
\NormalTok{plotbyrpsupport}
\end{Highlighting}
\end{Shaded}

\begin{verbatim}
## Warning: Removed 5 rows containing missing values (position_stack).
\end{verbatim}

\includegraphics{01_analysisRMDforknit_files/figure-latex/unnamed-chunk-21-1.pdf}

\#Fourth, I also do the same for the variable that tracks \% of
respondents who support democracy

\begin{Shaded}
\begin{Highlighting}[]
\NormalTok{plotbydemsupport <{-}}\StringTok{ }\KeywordTok{ggplot}\NormalTok{(}\DataTypeTok{data =}\NormalTok{ provdata, }\KeywordTok{aes}\NormalTok{(}\DataTypeTok{x =} \KeywordTok{reorder}\NormalTok{(ï..province, percent\_demsupport2008), }\DataTypeTok{y =}\NormalTok{ percent\_demsupport2008, }\DataTypeTok{by =}\NormalTok{ percent\_demsupport2008, }\DataTypeTok{fill =}\NormalTok{ Defections, }\DataTypeTok{color =}\NormalTok{ Defections)) }\OperatorTok{+}\StringTok{ }\KeywordTok{geom\_bar}\NormalTok{(}\DataTypeTok{colour=}\StringTok{"white"}\NormalTok{, }\DataTypeTok{stat=}\StringTok{"identity"}\NormalTok{) }\OperatorTok{+}\StringTok{ }\KeywordTok{labs}\NormalTok{(}\DataTypeTok{title=} \StringTok{"Levels of Support for Democracy by Province"}\NormalTok{, }\DataTypeTok{x=}\StringTok{"Province"}\NormalTok{, }\DataTypeTok{y=}\StringTok{"\% of Respondents Who Support Democracy"}\NormalTok{) }\OperatorTok{+}\StringTok{ }\KeywordTok{coord\_flip}\NormalTok{() }\OperatorTok{+}\StringTok{ }\KeywordTok{scale\_y\_continuous}\NormalTok{(}\DataTypeTok{breaks =} \KeywordTok{c}\NormalTok{(}\DecValTok{5}\NormalTok{, }\DecValTok{10}\NormalTok{, }\DecValTok{15}\NormalTok{, }\DecValTok{20}\NormalTok{, }\DecValTok{25}\NormalTok{, }\DecValTok{30}\NormalTok{, }\DecValTok{35}\NormalTok{, }\DecValTok{40}\NormalTok{, }\DecValTok{45}\NormalTok{, }\DecValTok{50}\NormalTok{, }\DecValTok{55}\NormalTok{, }\DecValTok{60}\NormalTok{, }\DecValTok{65}\NormalTok{, }\DecValTok{70}\NormalTok{, }\DecValTok{75}\NormalTok{, }\DecValTok{80}\NormalTok{, }\DecValTok{85}\NormalTok{, }\DecValTok{90}\NormalTok{))}
\NormalTok{plotbydemsupport}
\end{Highlighting}
\end{Shaded}

\begin{verbatim}
## Warning: Removed 5 rows containing missing values (position_stack).
\end{verbatim}

\includegraphics{01_analysisRMDforknit_files/figure-latex/unnamed-chunk-22-1.pdf}

\#Finally, I also create a boxplot that shows the divide between urban
and rural areas. Here we see a pretty big bump in the \# of defectors if
the province is largely urban.

\begin{Shaded}
\begin{Highlighting}[]
\NormalTok{plotbyurban <{-}}\StringTok{ }\KeywordTok{ggplot}\NormalTok{(provdata, }\DataTypeTok{fill =} \StringTok{"Defectors"}\NormalTok{) }\OperatorTok{+}\StringTok{ }\KeywordTok{geom\_boxplot}\NormalTok{(}\KeywordTok{aes}\NormalTok{(urban, defector\_number, }\DataTypeTok{group =}\NormalTok{ urban, }\DataTypeTok{fill =} \StringTok{"Defectors"}\NormalTok{)) }\OperatorTok{+}\StringTok{ }\KeywordTok{labs}\NormalTok{(}\DataTypeTok{title=}\StringTok{"Defectors by Urban/Rural, Jan 2014"}\NormalTok{, }\DataTypeTok{x=}\StringTok{"Urban/Rural"}\NormalTok{, }\DataTypeTok{y=}\StringTok{"Number of Defectors"}\NormalTok{) }\OperatorTok{+}\StringTok{ }\KeywordTok{scale\_y\_continuous}\NormalTok{(}\DataTypeTok{breaks =} \KeywordTok{c}\NormalTok{(}\DecValTok{1}\OperatorTok{:}\DecValTok{10}\NormalTok{)) }\OperatorTok{+}\StringTok{ }\KeywordTok{scale\_x\_continuous}\NormalTok{(}\DataTypeTok{breaks =} \DecValTok{0}\OperatorTok{:}\DecValTok{1}\NormalTok{)}
\NormalTok{plotbyurban}
\end{Highlighting}
\end{Shaded}

\includegraphics{01_analysisRMDforknit_files/figure-latex/unnamed-chunk-23-1.pdf}

\#What have we learned? Defectors seem to come largely from ethnic
Mossi, Moore-speaking, and urban provinces, but there is no obvious
correlation between local levels of support for democracy or disapproval
of the ruling party and defections. \#In other words, it is not clear
whether defectors are ``reflecting'' the attitudes of the population in
their home provinces (with the exception of maybe Kadiogo (where the
capital Ouagadougou is located), the outlier) \#To test the significance
of these variables, I ran a series of simply linear regressions (on both
\# of defectors and a binary of whether or not there were defections in
the province)

\begin{Shaded}
\begin{Highlighting}[]
\KeywordTok{lm}\NormalTok{(provdata}\OperatorTok{$}\NormalTok{defector\_number }\OperatorTok{\textasciitilde{}}\StringTok{ }\NormalTok{provdata}\OperatorTok{$}\NormalTok{percent\_donottrust2008)}
\end{Highlighting}
\end{Shaded}

\begin{verbatim}
## 
## Call:
## lm(formula = provdata$defector_number ~ provdata$percent_donottrust2008)
## 
## Coefficients:
##                     (Intercept)  provdata$percent_donottrust2008  
##                         0.07679                          0.04837
\end{verbatim}

\begin{Shaded}
\begin{Highlighting}[]
\KeywordTok{lm}\NormalTok{(provdata}\OperatorTok{$}\NormalTok{binary\_defector }\OperatorTok{\textasciitilde{}}\StringTok{ }\NormalTok{provdata}\OperatorTok{$}\NormalTok{percent\_donottrust2008)}
\end{Highlighting}
\end{Shaded}

\begin{verbatim}
## 
## Call:
## lm(formula = provdata$binary_defector ~ provdata$percent_donottrust2008)
## 
## Coefficients:
##                     (Intercept)  provdata$percent_donottrust2008  
##                        0.452724                         0.008259
\end{verbatim}

\begin{Shaded}
\begin{Highlighting}[]
\CommentTok{\#Note that the relationship is also not significant if we test by predominant ethnicity or language, whether the province was primarily urban/rural, or based on \% who support democracy in the province.}
\KeywordTok{lm}\NormalTok{(provdata}\OperatorTok{$}\NormalTok{defector\_number }\OperatorTok{\textasciitilde{}}\StringTok{ }\NormalTok{provdata}\OperatorTok{$}\NormalTok{primary\_ethnicity)}
\end{Highlighting}
\end{Shaded}

\begin{verbatim}
## 
## Call:
## lm(formula = provdata$defector_number ~ provdata$primary_ethnicity)
## 
## Coefficients:
##                          (Intercept)  
##                            1.500e+00  
##       provdata$primary_ethnicityBobo  
##                           -1.000e+00  
##     provdata$primary_ethnicityDagari  
##                            2.000e+00  
## provdata$primary_ethnicityGourmatche  
##                           -8.333e-01  
##  provdata$primary_ethnicityGourounsi  
##                           -1.000e+00  
##   provdata$primary_ethnicityKaraboro  
##                            5.000e-01  
##       provdata$primary_ethnicityLobi  
##                            1.000e+00  
##      provdata$primary_ethnicityMarka  
##                           -1.500e+00  
##      provdata$primary_ethnicityMossi  
##                            3.636e-01  
##       provdata$primary_ethnicityPeul  
##                           -1.000e+00  
##       provdata$primary_ethnicitySamo  
##                           -1.500e+00  
##    provdata$primary_ethnicitySenoufo  
##                            3.365e-15  
##     provdata$primary_ethnicitySonrai  
##                           -1.500e+00  
##      provdata$primary_ethnicityYanan  
##                            5.000e-01
\end{verbatim}

\begin{Shaded}
\begin{Highlighting}[]
\KeywordTok{lm}\NormalTok{(provdata}\OperatorTok{$}\NormalTok{defector\_number }\OperatorTok{\textasciitilde{}}\StringTok{ }\NormalTok{provdata}\OperatorTok{$}\NormalTok{primary\_language)}
\end{Highlighting}
\end{Shaded}

\begin{verbatim}
## 
## Call:
## lm(formula = provdata$defector_number ~ provdata$primary_language)
## 
## Coefficients:
##                        (Intercept)       provdata$primary_languageBobo  
##                          1.500e+00                          -1.000e+00  
##    provdata$primary_languageDagari    provdata$primary_languageFulfude  
##                          2.000e+00                          -1.000e+00  
## provdata$primary_languageGourounsi  provdata$primary_languageGulmacera  
##                         -1.000e+00                          -8.333e-01  
##  provdata$primary_languageKaraboro       provdata$primary_languageLobi  
##                          5.000e-01                           1.000e+00  
##     provdata$primary_languageMarka      provdata$primary_languageMoore  
##                         -1.500e+00                           3.636e-01  
##      provdata$primary_languageSamo    provdata$primary_languageSenoufo  
##                         -1.500e+00                           2.259e-15  
##    provdata$primary_languageSonrai      provdata$primary_languageYanan  
##                         -1.500e+00                           5.000e-01
\end{verbatim}

\begin{Shaded}
\begin{Highlighting}[]
\KeywordTok{lm}\NormalTok{(provdata}\OperatorTok{$}\NormalTok{defector\_number }\OperatorTok{\textasciitilde{}}\StringTok{ }\NormalTok{provdata}\OperatorTok{$}\NormalTok{urban)}
\end{Highlighting}
\end{Shaded}

\begin{verbatim}
## 
## Call:
## lm(formula = provdata$defector_number ~ provdata$urban)
## 
## Coefficients:
##    (Intercept)  provdata$urban  
##          1.025           4.375
\end{verbatim}

\begin{Shaded}
\begin{Highlighting}[]
\KeywordTok{lm}\NormalTok{(provdata}\OperatorTok{$}\NormalTok{defector\_number }\OperatorTok{\textasciitilde{}}\StringTok{ }\NormalTok{provdata}\OperatorTok{$}\NormalTok{percent\_demsupport2008)}
\end{Highlighting}
\end{Shaded}

\begin{verbatim}
## 
## Call:
## lm(formula = provdata$defector_number ~ provdata$percent_demsupport2008)
## 
## Coefficients:
##                     (Intercept)  provdata$percent_demsupport2008  
##                         2.28580                         -0.01248
\end{verbatim}

\begin{Shaded}
\begin{Highlighting}[]
\KeywordTok{lm}\NormalTok{(provdata}\OperatorTok{$}\NormalTok{binary\_defector }\OperatorTok{\textasciitilde{}}\StringTok{ }\NormalTok{provdata}\OperatorTok{$}\NormalTok{percent\_demsupport2008)}
\end{Highlighting}
\end{Shaded}

\begin{verbatim}
## 
## Call:
## lm(formula = provdata$binary_defector ~ provdata$percent_demsupport2008)
## 
## Coefficients:
##                     (Intercept)  provdata$percent_demsupport2008  
##                        1.307459                        -0.009963
\end{verbatim}

\#None are significant (although of course I didn't spend much time on
the specifications)

\hypertarget{section-3}{%
\section{----}\label{section-3}}


\end{document}
